\documentclass{article}
\usepackage[utf8]{inputenc}
\usepackage[T1]{fontenc}
\usepackage{amsmath}
\usepackage{amssymb}
\usepackage{listings}
\usepackage[usenames,dvipsnames]{color}

\lstset{%,
  basicstyle=\small\ttfamily,
  keywordstyle=\color{red},
  identifierstyle=\color{Blue},
  stringstyle=\color{OliveGreen},
  showstringspaces=false,
  language=Python}%,
%  basicstyle=\small\texttt}

\newcommand{\diff}[2]{\frac{\mathrm d #1}{\mathrm d #2}}
\newcommand{\Ha}[0]{\mathrm{Hartree}}
\newcommand{\XC}[0]{\mathrm{XC}}
\newcommand{\ext}[0]{\mathrm{ext}}

\title{Python for scientific scripting --- }
\author{Ask Hjorth Larsen}

\begin{document}

\section{Towards a simple 1D DFT code}

Our goal is to write our own Kohn--Sham (KS) density functional theory
(DFT) code.


A full-featured DFT code is very complex, so we shall here limit our ambitions
to the simplest possible model that is still interesting:
We will iteratively solve the Kohn--Sham
equations for a harmonic oscillator including electronic kinetic
energy, electrostatic repulsion between the electrons, and the
local density approximation for electronic interactions, ignoring correlation.

This gives us the full Hamiltonian
\begin{align}
\hat H = -\frac12 \diff{^2}{x^2} + v_\Ha(x) + v_{\mathrm X}^{\mathrm{LDA}}(x) + x^2.
\end{align}

Overview: We must be able to calculate the KS wavefunctions, the
density, and each of the potentials required to represent the
Hamiltonian.  We must also represent the Hamiltonian somehow,
including the kinetic operator.  But one thing at a time.

\section{Python}

Python is a dynamically typed language.  Python programs are executed
by the Python interpreter.

There are two main versions:
\texttt{python2} (also called just \texttt{python}) and \texttt{python3}.
Either is fine, but we use \texttt{python3}.
Run the interactive Python interpreter:

\begin{lstlisting}
askhl@hagen:~$ python3
Python 3.5.2 (default, Nov 17 2016, 17:05:23)
[GCC 5.4.0 20160609] on linux
Type "help", "copyright", "credits" or "license" for more information.
>>> print('hello world!')
hello world!
>>> 2+2
4
>>> items = [1, 2, 3, 'hello']
>>> for obj in items:
...     print(obj)
...
1
2
3
hello
>>>
\end{lstlisting}

Alternatively we can write code into a script:
\begin{lstlisting}
print('hello world!')
\end{lstlisting}
and save the script as \texttt{hello.py}.  Then we run the script using
\texttt{python3 hello.py}.

Although what you want to write is a script, be sure to use
the interactive interpreter to test things and play
around, so you know that the code does what you think it does.

Use \lstinline{help(obj)} to see the documentation for any object
(including functions and modules).

\section{Grids}

The simplest possible way to represent a real function $f(x)$,
with $a \le x \le b$, is to
sample it on a uniform real-space grid of points $\{x_i\}$ from $a$ to $b$.
The function is then represented by the values $\{f(x_i)\}$.

Use matplotlib to plot $\sin(x)$ on a suitable grid; see below.
How do you calculate the first and second derivatives of a function from
its grid representation?  Verify using matplotlib.

Useful functions:

\begin{lstlisting}
import numpy as np
import matplotlib.pyplot as plt
x = np.linspace(-5, 5, 200)  # define grid
y = np.sin(x)
plt.plot(x, y)
plt.show()
\end{lstlisting}



\begin{itemize}
\item Run \texttt{python3} to get an interactive inter
\item Get help about a function: \texttt{help(np.linspace)}, \texttt{help(}
\item \texttt{import numpy as np}
\item \texttt{np.linspace} and other NumPy functions
\item \texttt{import matplotlib.pyplot as plt}
\item \texttt{plt.plot} 
\end{itemize}

\section{Free non-interacting electrons}

To get the Kohn--Sham wavefunctions we must diagonalize the
Hamiltonian, i.e., calculate its eigenvectors.  Hence we need a matrix
representation of the Hamiltonian.

If our grid has $N$ points, then all possible linear operators are
those represented by $N\times N$ matrices.  An operator $O$ is applied
to a state $psi(x)$ by left-multiplying it onto the state,
\texttt{Opsi = np.dot(O, psi)}.  How can we represent the kinetic
operator as a matrix?

How can we represent the kinetic operator as a matrix?  Its action on
a state (e.g. $\sin x$) must be to produce minus twice the
second-order derivative of the state.

Find the wavefunctions of the lowest 5 

\begin{align}
  \hat H = \hat T = -\frac12 \diff{^2}{x^2}
\end{align}


\end{document}
