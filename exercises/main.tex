\documentclass{article}
\usepackage[utf8]{inputenc}
\usepackage[T1]{fontenc}
\usepackage{amsmath}
\usepackage{amssymb}
\usepackage{listings}
\usepackage[usenames,dvipsnames]{color}

\lstset{%,
  basicstyle=\small\ttfamily,
  keywordstyle=\color{red},
  identifierstyle=\color{Blue},
  stringstyle=\color{OliveGreen},
  showstringspaces=false,
  language=Python}%,
%  basicstyle=\small\texttt}

\newcommand{\dee}[0]{\mathrm d}
\newcommand{\idee}[0]{\,\dee}
\newcommand{\diff}[2]{\frac{\dee #1}{\dee #2}}
\newcommand{\pdiff}[2]{\frac{\partial #1}{\partial #2}}
\newcommand{\Ha}[0]{\mathrm{Hartree}}
\newcommand{\XC}[0]{\mathrm{XC}}
\newcommand{\X}[0]{\mathrm{X}}
\newcommand{\LDA}[0]{\mathrm{LDA}}
\newcommand{\ext}[0]{\mathrm{ext}}

\title{Python for scientific scripting --- }
\author{Ask Hjorth Larsen}

\begin{document}

\section{Towards a simple 1D DFT code}

Our goal is to write our own Kohn--Sham (KS) density functional theory
(DFT) code.


A full-featured DFT code is very complex, so we shall here limit our ambitions
to the simplest possible model that is still interesting:
We will iteratively solve the Kohn--Sham
equations for a harmonic oscillator including electronic kinetic
energy, electrostatic repulsion between the electrons, and the
local density approximation for electronic interactions, ignoring correlation.

This gives us the full Hamiltonian
\begin{align}
\hat H = -\frac12 \diff{^2}{x^2} + v_\Ha(x) + v_{\mathrm X}^{\mathrm{LDA}}(x) + x^2.
\end{align}
Overview: We must be able to calculate the KS wavefunctions, the
density, and each of the potentials required to represent the
Hamiltonian.  We must also represent the Hamiltonian somehow,
including the kinetic operator.  But one thing at a time.

\section{Python}
Python is a dynamically typed language.  Python programs are executed
by the Python interpreter.
There are two main versions:
\texttt{python2} (also called just \texttt{python}) and \texttt{python3}.
Either is fine, but we use \texttt{python3}.
Run the interactive Python interpreter:

\begin{verbatim}
askhl@hagen:~$ python3
Python 3.5.2 (default, Nov 17 2016, 17:05:23)
[GCC 5.4.0 20160609] on linux
Type "help", "copyright", "credits" or "license" for more information.
>>> print('hello world!')
hello world!
>>> 2+2
4
>>> items = [1, 2, 3, 'hello']
>>> for obj in items:
...     print(obj)
...
1
2
3
hello
>>>
\end{verbatim}

Alternatively we can write code in a text editor:
\begin{lstlisting}
print('hello world!')
\end{lstlisting}
Then save the script as \texttt{hello.py}.  Then we can run the script using
\texttt{python3 hello.py}.

Although you want to write a script, be sure to use
the interactive interpreter to play around and test things ---
then you know for sure that the code does what you think it does.

Use \lstinline{help(obj)} to see the documentation for any object
(including functions and modules).

\section{Grids}
The simplest possible way to represent a real function $f(x)$,
with $a \le x \le b$, is to
sample it on a uniform real-space grid of points $\{x_i\}$ from $a$ to $b$.
The function is then represented by the values $\{f(x_i)\}$.

Use matplotlib to plot $\sin(x)$ on a suitable grid; see below.  How
do you approximate the first and second derivatives of a function from
its grid representation?  Verify using matplotlib that your
derivatives are correct.

Useful functions:

\begin{lstlisting}
import numpy as np
import matplotlib.pyplot as plt
x = np.linspace(-5, 5, 200)  # define grid
y = np.sin(x)
plt.plot(x, y)
plt.show()
\end{lstlisting}



\begin{itemize}
\item Run \texttt{python3} to get an interactive inter
\item Get help about a function: \texttt{help(np.linspace)}, \texttt{help(}
\item \texttt{import numpy as np}
\item \texttt{np.linspace} and other NumPy functions
\item \texttt{import matplotlib.pyplot as plt}
\item \texttt{plt.plot}
\end{itemize}

\section{Free non-interacting electrons}

To get the Kohn--Sham wavefunctions we must diagonalize the
Hamiltonian, i.e., calculate its eigenvectors.  Hence we need a matrix
representation of the Hamiltonian.  We start with the kinetic operator
$\hat T$.

If our grid has $N$ points, then all possible linear operators on that
space are represented by $N\times N$ matrices.  An operator $O$ is
applied to a state $\psi(x)$ by left-multiplying it onto the state:
\texttt{Opsi = np.dot(O, psi)}.

How can we represent the action of the kinetic operator (which
includes a second derivative) as a matrix?  Use the expression
for the second derivative derived earlier.

Now we have a matrix representation of the kinetic operator; this
is the Hamiltonian of non-interacting free particles in a box given by
the size of our grid:
\begin{align}
  \hat H = \hat T = -\frac12 \diff{^2}{x^2}
\end{align}
The Kohn--Sham states and eigenvalues are eigenvectors and
eigenvalues of that matrix.  The matrix is real and symmetric, so we can use
\texttt{np.linalg.eigh}.

Plot the wavefunctions $\psi_n(x)$ with the lowest 5 energies.  What are the energies?

\section{Harmonic oscillator}
Now we include the external potential $v_{\mathrm{ext}}(x) = x^2$ in the Hamiltonian:
\begin{align}
  \hat H = \hat T = -\frac12 \diff{^2}{x^2} + x^2.
\end{align}
This is the harmonic oscillator for non-interacting particles.
How can you represent $x^2$ as a matrix?

Calculate the Hamiltonian and plot the 5 states with lowest energy,
making sure that your grid represents the solution reasonably well.

\section{Density}
Next we need to calculate the electron density.

Each state should be normalized so it integrates to one (having the
capacity to contain one electron):
\begin{align}
  \int |\psi(x)|^2 \idee x = 1.
\end{align}
The normalization from \texttt{np.linalg.eigh} will be different.
How do you calculate an integral over the grid?
Make sure all the states integrate to 1.

Then the electron density in DFT is given by
\begin{align}
  n(x) = \sum_n f_n |\psi_n(x)|^2,
\end{align}
where $f_n$ are the \emph{occupation numbers}.

The electrons will
preferentially occupy those available states with lowest energy.  Each
state fits two electrons (one with spin up, and one with spin down), so
a state can in general be occupied by 0, 1, or 2 electrons.

Let us say that we have 6 electrons.  Then the three lowest states
will have occupation number $f=2$ and all others $f=0$.

Plot the electron density for 6 electrons in the harmonic potential.
Verify that the density integrates to 6 electrons.

\section{Exchange energy}
The exchange (and correlation) energy is a correction to the
electronic energy that approximates the effect of electron interactions.
We considerhere only the exchange which is particularly simple in
the local density approximation, LDA:
\begin{align}
  E_\X^\LDA[n] = \int n^{4/3} \idee
\end{align}


\section{Self-consistency loop}
Write a loop

\section{Coulomb potential}
The electrostatic energy or Hartree energy is given by
\begin{align}
  E_{\mathrm{Ha}}^{\mathrm{3D}} = \frac12 \iint \frac{n(\mathbf r)n(\mathbf r')}{|\mathbf r - \mathbf r'|}\idee \mathbf r \idee \mathbf r'.
\end{align}
This expression converges in 3D, but not in 1D.  Hence we cheat and use
a modified, ``softened'' form:
\begin{align}
E_{\mathrm{Ha}} = \frac12 \iint \frac{n(x) n(x')}{\sqrt{(x - x')^2 + 1}}.
\idee x \idee x'
\end{align}
Calculate the (soft) Coulomb energy of the density from before.

In the derivation of the Kohn--Sham equations, the potential is defined
as the derivative of the total energy with respect to the density.
The Coulomb \emph{potential} is therefore given by the derivative
of the Coulomb energy with respect to the density:

\begin{align}
  v_{\mathrm{Ha}}(x) = \pdiff{E_{\mathrm{Ha}}}{n(x)} =
  \int \frac{n(x)}{\sqrt{(x - x')^2 + 1}} \idee x
\end{align}

Calculate the Coulomb potential.


\end{document}
