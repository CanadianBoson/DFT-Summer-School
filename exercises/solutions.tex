\documentclass{article}
\usepackage[utf8]{inputenc}
\usepackage[T1]{fontenc}
\usepackage{amsmath}
\usepackage{amssymb}
\usepackage{listings}
\usepackage[usenames,dvipsnames]{color}
\usepackage{boxedminipage}
\usepackage{graphicx}

\lstset{%,
  basicstyle=\small\ttfamily,
  keywordstyle=\color{red},
  identifierstyle=\color{Blue},
  stringstyle=\color{Orange},
  commentstyle=\ttfamily\color{Green},
  showstringspaces=false,
  language=Python}%,
%  basicstyle=\small\texttt}


\newcommand{\exercise}[1]{
  \bigskip
  \begin{boxedminipage}[c]{0.8\linewidth}
  #1
  \end{boxedminipage}\bigskip\\
}

\newcommand{\dee}[0]{\mathrm d}
\newcommand{\idee}[0]{\,\dee}
\newcommand{\diff}[2]{\frac{\dee #1}{\dee #2}}
\newcommand{\pdiff}[2]{\frac{\partial #1}{\partial #2}}
\newcommand{\Ha}[0]{\mathrm{Ha}}
\newcommand{\XC}[0]{\mathrm{XC}}
\newcommand{\X}[0]{\mathrm{X}}
\newcommand{\LDA}[0]{\mathrm{LDA}}
\newcommand{\ext}[0]{\mathrm{ext}}


\begin{document}
\section{Derivatives}
By definition,
\begin{align}
  f'(x) = \lim_{\delta\rightarrow 0} \frac{f(x + h) - f(x)}{h}.
\end{align}
On a uniform grid $\{x_i\}$, the smallest distance we can represent is the grid
spacing.  A natural choice is therefore the following difference quotient:
\begin{align}
  f'(x) \approx \frac{f(x + h) - f(x)}{h},
\end{align}
where $h$ is the grid spacing.
But this is a \emph{left} derivative: It uses $x$ and $x + h$ to
approximate the derivative at $x$.  Intuition should tell us (right?) that
it actually best approximates the derivative \emph{between} those two points,
i.e., $f'(x + h/2)$, which is not part of our grid.
If we are going to do computations with both $f$ and $f'$, we should wisely
try to have them on the same grid so we can easily do arithmetic with them.
We therefore write:
%\begin{align}
%  f'(x) \approx \frac{f(x + h/2) - f(x - h/2)}{h},
%\end{align}
%but then we get the derivative on a new grid, displaced by half a grid
%spacing, $h/2$.  That would make it inconvenient if/when we need to
%multiply the function by its derivative.  This is a better
%approximation which also has the property that calculated derivatives
%$f'$ remain on the same grid:
\begin{align}
  f'(x) \approx \frac{f(x + h) - f(x - h)}{2h}.
\end{align}
We say that this is a \emph{central} finite-difference
derivative since the function values are symmetric around the point where we
calculate the derivative.
To get the second-order derivative, we apply this expression twice:
\begin{align}
  f''(x) &\approx \frac{1}{2h} \left[f'(x + h) - f'(x - h) \right] \nonumber\\
  &= \frac{1}{4h^2} \left[
  f(x + 2 h) - 2 f(x) + f(x - 2 h)
  \right].
\end{align}
In this expression we only see differences of $2h$.   Hence we take $2h$ to
be the grid spacing and rewrite accordingly:
\begin{align}
  f''(x) \approx \frac{1}{h^2}\left[f(x + h) - 2 f(x) + f(x - h)\right].
\end{align}
In conclusion, this is how we would calculate the second-order
derivative from function values on a grid.

One can use a higher-order Taylor expansion and obtain expressions that
involve several other ``nearest neighbours'': $f(x), f(x \pm h), f(x \pm
2h), \ldots$.  Such expressions give higher accuracy if the grid is fine
enough.


How can we represent the kinetic operator as a matrix?
Note how the expression for the second derivative is a linear combination
of function values at different (neighbouring) grid points.
We arrange the coefficients $-2$ and $+1$ in the diagonal and the first off-diagonals.
Then it is straightforward to verify that:
\begin{align}
  \mathbf T \mathbf y &=
  -\frac{1}{2h^2} \left[
    \begin{matrix}
      -2 & \hphantom +1  & \hphantom + 0 & &\cdots& \hphantom + 0\\
      \hphantom +1  & -2 & \hphantom +1 &       && \vdots\\
      %0  & 1  & -2 & 1 & \\
      \hphantom + 0 & \hphantom +1 & -2 &  &&  \\
      %\vdots & & 1 & -2 & 1\\
       & & & \ddots&   & \hphantom + 0\\
      \vdots & & & &  -2& \hphantom +1\\
      \hphantom + 0 & \cdots && \hphantom + 0 & \hphantom +1 & -2
    \end{matrix}
    \right]
  \left[
    \begin{matrix}
    f(x_1)\\
    \vdots\\
    f(x_i)\\
    \vdots\\
    f(x_N)\\
    \end{matrix}
    \right]
  \nonumber\\
  &= -\frac{1}{2h^2} \left[
    \begin{matrix}
    \vdots\\
    f(x_i - h) - 2 f(x_i) + f(x_i + h)\\
    \vdots
    \end{matrix}
    \right].
\end{align}

\begin{figure}
  \includegraphics[scale=0.5]{script/derivatives.pdf}
  \caption{Derivatives calculated by finite differences, and action of
    the kinetic operator $\hat T$.}\label{fig:deriv}
\end{figure}


\noindent All the derivatives are shown on Figure \ref{fig:deriv}.
The script which calculates and plots them is this:
\lstinputlisting{script/ex_fd_T.py}

\section{Free particles and the harmonic oscillator}
If we simply take $\mathbf T$ to be the whole Hamiltonian, we are
non-interacting particles within a box as large as our grid.
We get the independent-particle wavefunctions using this script:
\lstinline{independent_particles.py}
This gives the wavefunctions shown on Figure \ref{fig:box}.

\begin{figure}
  \includegraphics[scale=0.5]{script/particles_in_box.pdf}
  \caption{Particles in a box.}
  \label{fig:box}
\end{figure}

The second part of the above listed script adds a quadratic potential
to obtain the wavefunctions for the harmonic oscillator, shown on Figure
\ref{fig:harmonic}
\begin{figure}
  \includegraphics[scale=0.5]{script/harmonic_oscillator.pdf}
  \caption{Harmonic oscillator, whose exact energies are $1/2, 3/2,
    5/2, 7/2, \ldots$.  The calculated values are slightly off due
  to the finite precision of the grid.}
  \label{fig:harmonic}
\end{figure}

Finally we need to implement the different potentials and a
self-consistency loop.

\lstinputlisting{script/dft.py}

\end{document}
